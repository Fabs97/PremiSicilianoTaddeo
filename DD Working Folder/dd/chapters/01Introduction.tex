\defChapterTarget{Introduction}
    \section{Purpose}
    The purpose of this document is to give more technical details than the RASD
    about SafeStreets system. The RASD presented a more abstract and general
    view of the system and of the functions is supposed to execute. Indeed, this
    document presents more details about the design, run-time processes,
    deployment and algorithm. It also provideds more information about
    implementation, integration and testing with a testing plan.\\
    In particular, the document presents the following topics:
    \begin{itemize}
        \item Overview of the high-level architecture
        \item The main components, their interfaces and deployment
        \item The run-time behavior
        \item The desing patterns
        \item Requirements on architecture components
        \item Implementation plan
        \item Integration plan
        \item Testing plan
    \end{itemize}    
    
    \section{Scope}
    Here it's presented a review of the application scope, made referring to
    what has been stated in RASD document.\\ With SafeStreets users can notify
    the authorities when traffic violations occur, and in particular parking
    violations. Both user and authorities must register to the application and
    agree that SafeStreets stores the information provided, completing it with
    suitable meta-data. The whole system, because it tracks users information,
    must respect the standards defined for processing of sensitive information
    such as GDPR if it is used in Europe. The user sends the type of the
    violation to the municipality and direct proofs of it (like a photograph).
    The system runs an algorithm to read the license plate and also asks the
    user to directly insert the license for a better recognition. Obviously,
    other information are required, like the name of the street when the
    violation has occurred, which can be retrieved from user's direct input or
    from the geographical position of the violation (using Google Maps API).
    Furthermore, the system, by cross referencing data from third party
    services, automatically can highlight the streets with the highest frequency
    of violations or the vehicles that commit the most violations. SafeStreets
    crosses information about the accidents that occur on the territory of the
    municipality with his own data to identify potentially unsafe areas and
    suggest possible interventions. Because municipality could generates traffic
    tickets from the information about violations SafeStreets should guarantee
    that information is never altered (if a manipulations occurs, the
    application should discard the information). Such features are made possible
    trough the use of one mobile application with two different UIs which are
    determined by the kind of customer that logs in (user or PO). The collected
    information are sent to a back-end and they all of those can be accessed by
    municipality employees in order to execute different actions (emit ticket,
    analyze unsafe areas, etc...).
    
    \section{Definitions, Acronyms, Abbreviations}
        \subsection{Definitions}
        {User:} it is identified as a civilian customer of the product. It will
        be the main source for the SafeStreets initiative to obtain information
        about traffic violations and therefore to be successful;
        \paragraph{Third parties:}those kind of organization/company that could
        provide services useful to SafeStreets; \paragraph{Customer:} it defines
        both authorithy users (police officers or municipality employees) and
        civilians; \paragraph{Authority user:}all of those customers who have a
        responsibility role in regard of the streets' safety and the SafeStreets
        initiative. Example of these category are: police officers, municipal
        employees, director and basically anyone in charge and able to issue
        fines and deal with road violations; \paragraph{Ghiro:} image
        manipulation detection software, used by authorithy users in order to
        detect any image manipulation and assess the veracity of the hard
        evidence connected to the traffic ticket\\
        \subsection{Acronyms}
        \paragraph{UI:} User Interface \paragraph{GDPR:} General Data Protection
        Regulation \paragraph{API:} Application Programming Interface
        \paragraph{GPS:} Global Positioning System \paragraph{PO:} Police
        Officer \paragraph{ME:} Municipality Employee
        \subsection{Abbreviations}
        \paragraph{Gn:} nth goal; \paragraph{Dn:} nth assumption;
        \paragraph{Rn:} nth requirement; \paragraph{ID: } identifier (Fiscal
        Code for Users, a municipality identifier for Authorithy Users)
        
    \section{Revision history}
    \section{Reference Documents}
    \section{Document Structure}
        \subsection*{\chapIntroduction[Chapter 1 - Introduction]}
        Chapter 1 is an introduction of the design document. It describes the
        purpose and the scope of the document and it highlights the differences
        with the RASD. \\It also shows some abbreviations, definitions and
        acronyms in order to provide a better understanding of the document to
        the reader.

        \subsection*{\chapArchitecturalDesign[Chapter 2 - Architectural design]}
        Chapter 2 deals with the architectural design of the system and it's the
        core section of the document. \\
        It provides an overview of the architecture and it contains the most
        relevant architecture views:
        \begin{itemize}
            \item Component view
            \item Deployment view
            \item Run-time view
        \end{itemize}
        It also shows the interaction of the interfaces and the selected
        architectural styles and patterns, with an explanations of each one of
        them.
                
        \subsection*{\chapUserInterfaceDesign[Chapter 3 - User interface design]}
        Chapter 3 specifies the user interface design and refers to the mock-ups
        already presented in the RASD.

        \subsection*{\chapRequirementsTraceability[Chapter 4 - Requirements traceability]}
        Chapter 4 explains how the requirements defined in the RASD map to the
        design elements defined in this document.

        \subsection*{\chapImplementationIntegrationAndTestPlan[Chapter 5 - Implementation, integration and test plan]}
        Chapter 5 specifies the description and the order of implementation,
        integration and testing plan of the sub-components of the system.
        
        \subsection*{\chapEffortSpent[Chapter 6 - Effort spent]}
        Chapter 6 shows the effort spent by each member of the group working on
        this project.
    
        \subsection*{\chapReferences[Chapter 7 - References]}
        Chapter 7 includes the reference documcuments.
        