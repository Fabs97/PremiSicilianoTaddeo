\documentclass{article}
\usepackage{xparse}

\title{SafeStreets}
\author{Marco Premi, Fabrizio Siciliano, Giuseppe Taddeo}
\pagestyle{headings}

\begin{document}
\maketitle

\tableofcontents

\newpage
\section{Introduction}
\subsection{Purpose}
\subsubsection{General Purpose}
    The purpose of this document is to correctly analyze all requirements, goals 
    and actions needed in order to correctly develop SafeStreets.\\
    \\
    SafeStreets is a crowd-sourced application that intends to provide users with the possibility to notify authorities when
    traffic violations occur (eg. traffic violations) and will help to maintain stability and order within the streets. A reporting
    system will also be available to police offices and municipal employees in order to allow them to analyze (and take actions accordingly)
    different areas of the city and assess which areas have the most violations committed in.\\
    \\
    The core application will focus on storing useful traffic violation data provided by users, mainly with the help of input forms and
    hard evidence such as images. At any violation input, SafeStreets will also store useful metadata such as date and time the violation 
    was retrieved, geolocate where it is and update a city wide map highlighting the areas where violations happen.\\
    \\
    In addition to that, with the help of third parties (eg. municipality), SafeStreets will be able to retrieve the data given by such party
    and cross-reference them with its own data retrieved by users. By doing so, it will be possible to identify unsafe areas, assess which kind 
    of problems happen more frequently and suggest possible intervention. It will also be possible for third parties (municipality and police officers)
    to automatically generate traffic tickets. This will be happening in order to cross-reference all available data in order to build statistics such 
    as the most (or less) egregious offenders or the effectiveness of the SafeStreets initiative
    \subsubsection{Goals} Follows a list of all goals that will be reached with the SafeStreets initiative.
    \begin{itemize}
        \item G1: The system must allow all kind of users (both third parties and civilians) to correctly input (with hard evidence) traffic violations around the city;
        \item G2: The system must autonomously retrieve metadata from hard evidence useful to report and to all users;
        \item G3: The system must allow to create different clearance levels in order to offer different reporting systems to different kind of users;
        \item G4: The system will provide an efficient reporting system in order to highlight different violation categories through all parts of the city where the initiative is active;
    \end{itemize}
\subsection{Scope}
\subsection{Definitions, Acronyms, Abbreviations}
\subsubsection{Definitions}
\paragraph{User:} it is identified as a civilian customer of the product. It will be the main source for the SafeStreets initiative to obtain information about traffic violations and therefore be successful;
\paragraph{Third parties:}those kind of organization/company that could provide services useful to SafeStreets and that will be able to retrieve data in order to improve the streets' safety;
\subsubsection{Acronym}
\paragraph{UI:} User Interface
\paragraph{GDPR:} General Data Protection Regulation
\paragraph{API:} Application Programming Interface
\subsubsection{Abbreviations}
\paragraph{Gn:} nth goal;
\paragraph{Dn:} nth domain;
\paragraph{Rn:} nth requirement;
\subsection{Revision History}
\subsection{Reference Documents}
\subsection{Document Structure}

\newpage
\section{Overall Description}
\subsection{Product Perspective}
\subsection{Product Functions}
%qua vanno aggiunte le varie funzioni
\subsection{User characteristics}
\subsection{Assumptions, dependencies and constraints}
\subsubsection{Assumptions}
\subsubsection{Dependencies}
\subsubsection{Constraints}

\newpage
\section{Specific Requirements}
\subsection{External Interface Requirements}
\subsubsection{User Interfaces}
\subsubsection{Hardware Interfaces}
\subsubsection{Software Interfaces}
\subsubsection{Communication Interfaces}
\subsection{Scenarios}
\subsection{Functional requirements}
\subsection{Use Cases}
\subsubsection{User use cases}
\subsubsection{Third party use cases}
%aggiungo subsections
\subsection{Performance requirements}
\subsection{Design Constraints}
\subsubsection{Standars compliance}
\subsubsection{Hardware Limitations}
\subsubsection{Any Other Contraint}
\subsection{Software System Attributes}
\subsubsection{Reliability}
\subsubsection{Availability}
\subsubsection{Security}
\subsubsection{Mantainability}
\subsubsection{Portability}

\newpage
\section{Formal Analysis using Alloy}

\newpage
\section{Effort spent}
\begin{center}
    \begin{tabular}{c|c|c|c|c}
        \hline
        \textbf{Description of the task} & \textbf{MP} & \textbf{FS} & \textbf{GT} \\
        Introduction                    & 0 & 2 & 0 \\
        Overall Description             & 0 & 0 & 0 \\
        Specific requirements           & 0 & 0 & 0 \\
        Formal analysis using Alloy     & 0 & 0 & 0 \\
    \end{tabular}
\end{center}
\section{References}
    
\end{document}  