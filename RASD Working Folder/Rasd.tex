\documentclass{article}
\usepackage{xparse}

\title{SafeStreets}
\author{Marco Premi, Fabrizio Siciliano, Giuseppe Taddeo}
\pagestyle{headings}

\begin{document}
\maketitle

\tableofcontents

\newpage
\section{Introduction}
\subsection{Purpose}
\subsubsection{General Purpose}
    The purpose of this document is to correctly analyze all requirements, goals
    and actions needed in order to correctly develop SafeStreets.\\
    \\
    SafeStreets is a crowd-sourced application that intends to provide users
    with the possibility to notify authorities when traffic violations occur
    (eg. traffic violations) and will help to maintain stability and order
    within the streets. A reporting system will also be available to police
    offices and municipal employees in order to allow them to analyze (and take
    actions accordingly) different areas of the city and assess which areas have
    the most violations committed in.\\
    \\
    The core application will focus on storing useful traffic violation data
    provided by users, mainly with the help of input forms and hard evidence
    such as images. At any violation input, SafeStreets will also store useful
    metadata such as date and time the violation was retrieved, geolocate where
    it is and update a city wide map highlighting the areas where violations
    happen.\\
    \\
    In addition to that, with the help of third parties (eg. municipality),
    SafeStreets will be able to retrieve the data given by such party and
    cross-reference them with its own data retrieved by users. By doing so, it
    will be possible to identify unsafe areas, assess which kind of problems
    happen more frequently and suggest possible intervention. It will also be
    possible for third parties (municipality and police officers) to
    automatically generate traffic tickets. This will be happening in order to
    cross-reference all available data in order to build statistics such as the
    most (or less) egregious offenders or the effectiveness of the SafeStreets
    initiative \subsubsection{Goals} Follows a list of all goals that will be
    reached with the SafeStreets initiative.
    \begin{itemize}
        \item G1: The system must allow all kind of users (both third parties
        and civilians) to correctly input (with hard evidence) traffic
        violations around the city;
        \item G2: The system must autonomously retrieve metadata from hard
        evidence useful to report and to all users;
        \item G3: The system must allow to create different clearance levels in
        order to offer different reporting systems to different kind of users;
        \item G4: The system will provide an efficient reporting system in order
        to highlight different violation categories through all parts of the
        city where the initiative is active;
        \item G5: The system allows the authorities to be able to use the 
        application data to generates traffic tickets to civilians who commit 
        infractions (the requirement of authenticity of the chain of custody 
        follows);
    \end{itemize}
\subsection{Scope}
With SafeStreets users can notify the authorities when traffic violations occur,
and in particular parking violations. Both user and authorities must register to
the application and agree that SafeStreets stores the information provided,
completing it with suitable meta-data. The whole system, because it tracks users
information, must respect the standards defined for processing of sensitive
information such as GDPR if it is used in Europe. The user sends the type of the
violation to the municipality and direct proofs of it (like a photograph). The
system runs an algorithm to read the license plate and also asks the user to
directly insert the license for a better recognition. Of course other
information are required, like the name of the street when the violation has
occurred, which can be retrieved from user's direct input or from the
geographical position of the violation (using Google Maps API). Both users and
authorities can highlight the streets with the highest frequency of violations
or the vehicles that commit the most violations. SafeStreets crosses information
about the accidents that occur on the territory of the municipality with his own
data to identify potentially unsafe areas and suggest possible interventions.
Because municipality could generates traffic tickets from the information about
violations SafeStreets should guarantee that information is never altered (if a
manipulations occurs, the application should discard the information). Such
features are made possible trough the use of two mobile applications(one for the
citizens and one for the officers on the field). The collected information are
sent to a back-end. All the services can also be accessed through a specific
web-site. 
\subsection{Definitions, Acronyms, Abbreviations}
\subsubsection{Definitions}
\paragraph{User:} it is identified as a civilian customer of the product. It
will be the main source for the SafeStreets initiative to obtain information
about traffic violations and therefore be successful; \paragraph{Third
parties:}those kind of organization/company that could provide services useful
to SafeStreets and that will be able to retrieve data in order to improve the
streets' safety;
\subsubsection{Acronym}
\paragraph{UI:} User Interface \paragraph{GDPR:} General Data Protection
Regulation \paragraph{API:} Application Programming Interface
\subsubsection{Abbreviations}
\paragraph{Gn:} nth goal; \paragraph{Dn:} nth domain; \paragraph{Rn:} nth
requirement;
\subsection{Revision History}
\subsection{Reference Documents}
\subsection{Document Structure}
\paragraph{Chapter 1 - Introduction}
Gives an introduction to the problem by describing the purpose of SafeStreets.
It also shows the goals and the scope of the application. 
\paragraph{Chapter 2 - Overall Description}
Offers an overall description of the project. It identifies the actors involved
in the application and lists all the assumptions in order to identify all the
boundaries of the project. The product perspective includes details on the
shared phenomena and the domain models. The class diagram describe the domain
model used and the state diagrama analyzes:
\begin{itemize}
    \item The process of collecting violations from users
    \item The process of sharing informations with the municipality
\end{itemize}
The majority of functions of the system are more precisely specified by taking
in mind the goals of the system.  
\paragraph{Chapter 3 - Specific Requirements}
Contains external interface requirements which are: user interfaces, hardware
interfaces, software interfaces and communication interfaces. Few scenarios
describing how the system acts in real world are listed here. Furthermore it
provides the description of the functional requirements, through the use of use
cases and sequence diagrams. The non-functional requirements are defined through
performance requirements, design constraints and software system attributes.
\paragraph{Chapter 4 - Formal analysis using Alloy}
Includes the alloy model of some critical aspects with comments and
documentation.
\paragraph{Chapter 5 - Effort Spent}
Shows the effort spent by each single group member while working on the RASD.
\paragraph{Chapter 6 - References}
Includes the documents we used as reference.

\newpage
\section{Overall Description}
\subsection{Product Perspective}
afeStreets is designed to be a completely new software applications. It uses
some already proven services (like Google Maps and PlateRecognizer APIs) for its
critical tasks. The software uses these services in order to double check
whether both addresses and license plates are correctly standardized in order to
be stored into the violations database.\\
The system is composed of two different mobile applications: one for the
citizens that want to reports violations and one for the officers acting on the
field. It also provides a web site for third party users which allows them to
assess and analyze potential unsafe areas, thanks also to a powerful reporting
system.\\
Taken into consideration that the municipality could generate traffic tickets
from the input violations, the software will be critical when it comes to
handling chain of custody. The latter is assured to never be broken by not
allowing any kind of user to modify the reported violation. Supposedly, when
some traffic violations might be erroneous or do not have any reason of
existence, the inputting user can warn the responsible third party by attaching
a warning explaining why it should not be taken into consideration. The systems
also ensures the veracity of each violation and the hard evidence attached to it
by running an algorithm of image manipulation detection. This process is used by
third parties before the emission of each ticket to the corresponding offender.
%generico, poi aggiorniamo

\subsection{Product Functions}
In the following section the most important product functions of the system are 
reported.

    \subsubsection{Reporting traffic violations}
    This function is the main action that the user can use. After having
    logged in within its application, pressing on "Report Infraction" the system 
    will ask to insert an image specifying its format and characteristics (clear 
    image, defined context, visible plate). Once the image is loaded, the system 
    will ask him to enter the type of infraction reported, the date and time, as 
    well as the position detected by GPS signal or manually via user input. At this 
    point the user will confirm the data entered and save the report. Moreover he 
    will see in his personal area, under the list of his reports, the new report 
    with all the data he has entered. In addition, the report will be added to the 
    list of reports made in the same municipality.

    \subsubsection{Police Officer - Reportings Management (APP)}
    This function allows the police officer to have in real time all the reports 
    made in his municipality of competence. In fact, after logging into his 
    application and providing his identification as a police officer, the 
    application through an analysis system of the latter will identify the 
    municipality of competence of the agent and will show him the list of all the 
    appropriate reports. He can therefore go to the place of the report to resolve 
    the infraction manually by clicking "Resolve Live" and closing the infraction 
    or, in the event that the violation can no longer be assessed in person 
    (i.e. the car has moved) he can leave the report opened to then resolve it 
    from the website.

    \subsubsection{Police Officer - Reportings Management (WEB)}
    This function allows the police officer to resolve the reports (i.e. generating 
    a traffic ticket for the owner of the machine) via the website without being 
    physically present on the spot. Once logged in using codes provided by the 
    municipality, the police officer will be able to see all the reports still open 
    (i.e. no one has clicked "Resolve Live") and analyze them. He can take the photo 
    posted by the user and submit it to specialized programs to ensure that it is 
    not tampered with and will therefore be able to generate the ticket and close 
    the report.

    \subsubsection{Municipal Employee - Interventions Management}
    This function allows a municipal employee, after logging in through his own 
    identifier, to be able to view all the reports made in the municipality of 
    competence and be able to insert in the application the accident data that occur 
    in the territory. In this way, by pressing the "Start Analysis" button, SafeStreets 
    will identify potentially unsafe areas and show them to the municipal employee and 
    the municipal employee will be able to decide whether to take precautions 
    (i.e. insert barrier between cycle path and road).
    
\subsection{User characteristics}
\subsection{Assumptions, dependencies and constraints}
\subsubsection{Assumptions}
\subsubsection{Dependencies}
\subsubsection{Constraints}

\newpage
\section{Specific Requirements}
\subsection{External Interface Requirements}
\subsubsection{User Interfaces}
\subsubsection{Hardware Interfaces}
\subsubsection{Software Interfaces}
\subsubsection{Communication Interfaces}
\subsection{Scenarios}
\subsection{Functional requirements}
\subsection{Use Cases}
\subsubsection{User use cases}
\subsubsection{Third party use cases}

\subsection{Performance requirements}
\subsection{Design Constraints}
\subsubsection{Standars compliance}
\subsubsection{Hardware Limitations}
\subsubsection{Any Other Contraint}
\subsection{Software System Attributes}
\subsubsection{Reliability}
\subsubsection{Availability}
\subsubsection{Security}
\subsubsection{Mantainability}
\subsubsection{Portability}

\newpage
\section{Formal Analysis using Alloy}

\newpage
\section{Effort spent}
\begin{center}
    \begin{tabular}{c|c|c|c|c}
        \hline
        \textbf{Description of the task} & \textbf{MP} & \textbf{FS} &
        \textbf{GT} \\
        Introduction                    & 2.5   & 2     & 0     \\
        Overall Description             & 1.5   & 1.5   & 0     \\
        Specific requirements           & 0     & 0     & 0     \\
        Formal analysis using Alloy     & 0     & 0     & 0     \\
    \end{tabular}
\end{center}
\section{References}
    \paragraph{Plate Recognizer:} https://app.platerecognizer.com
    
\end{document}  